\section{Lesson01 Setting Up An OpenGL Window}

\begin{enumerate}
	\item OpenGL uses some new symbol to represent commonly used ones, for example, \verb|GLvoid| for \verb|void|, \verb|GLsizei| for \verb|int|, \verb|GLfloat| for \verb|float|.
	\item \verb|glViewport| is used to reset the current viewport.
	\item \verb|glMatrixMode| is used to select matrices which may later be modified.
	\item \verb|glLoadIdentity| is used to reset matrix.
	\item \verb|gluPerspective| is used to calculate the aspect ratio of the window.
	\item \verb|glShadeModel| is used to set shade model, the parameter could be \verb|GL_SMOOTH|, which indicates smooth shading.
	\item \verb|glClearColor| is used to set background color.
	\item \verb|glClearDepth| is used to clear depth buffer.
	\item \verb|glEnable| is used to enable some functions in OpenGL.
	\item \verb|glDepthFunc| is used to designate specific depth test function.
	\item \verb|glClear| can be used to clear the screen and the depth buffer.
\end{enumerate}

\section{Lesson02 Your First Polygon}
\begin{enumerate}
	\item The 3D coordinate system in OpenGL contains x-axis pointing from left to right, y-axis pointing from bottom to top and z-axis pointing from the back of screen to the front of screen.
	\item \verb|glTranslatef| is used to translate current drawing location relatively.
	\item \verb|glBegin| can be used to begin drawing. In this example, it is used to begin drawing triangles and quads with paramters \verb|GL_TRIANGLES| and \verb|GL_QUADS| respectively.
\end{enumerate}


\section{Outline Fonts}

It uses \verb|GLYPHMETRICSFLOAT| to store information about fonts.

It generates 256 display lists to 


